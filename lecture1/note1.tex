
%\section{Lecture 1: Private Information Retrieval}
The Private Information Retrieval problem was first introduced by Chor, Kushilevitz, Goldreich and Sudan~\cite{chor1998private}.
In this setting, we will have a client and 
one or more server(s). The servers each have a public database indexed from $1$ to $n$ (e.g., the DNS repository, 
a repository of webpages, a leaked password database, etc).

A client wants to fetch an entry 
indexed $i \in [n]$ 
from this database 
but does not want to leak its query to the server(s). More formally,
we define a single-server PIR scheme as follows.

\begin{definition}[Single-server PIR]
A single-server PIR, parametrized
by a security parameter $\lambda \in \mathbb{N}$, is a 
protocol between a client and a server with the following
syntax: 
\begin{itemize}
	\item The client's input is a desired 
index $i \in [n]$, and the server's input is a database $\DB \in \{0,1\}^n$.
Both the client and server also obtain $1^\lambda$ as input. 
	\item At the end of the protocol, the client outputs a bit $b \in \{0,1\}$.
\end{itemize}

We want the scheme to satisfy the following properties.
\begin{itemize}
\item \textbf{Correctness}: for all $\lambda, n$, 
%any $n$ that is polynomially bounded in $\lambda$, 
for any $\DB \in \{0,1\}^n, i \in [n],$ under honest execution, 
	\[\Pr[b = \DB[i]] = 1 %- \mu(n) %\tag{negligible $\mu$}
\]
	
\item \textbf{Privacy}: For any $\lambda$, 
any $n$ polynomially bounded in $\lambda$, any 
$i,j \in [n], \DB  \in \{0,1\}^n$, 
it holds that  
%and function $view_{s}$ representing the view of server $s$:
	\[{\sf view}_S(1^\lambda, \DB,i) \approx {\sf view}_S(1^\lambda, \DB,j)\]
where ${\sf view}_S(1^\lambda, \DB, i)$ 
is a random variable representing the view of the server  
if we execute the PIR protocol 
over client input $(1^\lambda, i)$ and server input $(1^\lambda, \DB)$, 
and $\approx$ stands for 
statistical or computational indistinguishability.
%identically distributed, statistically, 
%or computationally indistinguishable.
%representing the view of server $s$:
\end{itemize}
\end{definition}
%Intuitively, the privacy definition means that from the server's view, it cannot tell what the user is querying. One can also define a definition for malicious privacy, but since our schemes of interest are one-round, we do not make this distinction.

\begin{remark}[Honest-server vs. malicious-server privacy]
The above privacy definition assumes an honest server.
It is also possible to define privacy against a malicious server.
In today's lecture, all the PIR constructions
will only have a single round-trip --- 
in this special case, honest-server privacy
and malicious-server
privacy are equivalent. So we will simply define honest-server privacy here.
\end{remark}



This definition is naturally extended to 
a setting with two or more servers that do not communicate, 
where privacy should hold for any individual server's view. 
In a setting with more than two servers, it also makes
sense to define $t$-out-of-$n$ security, where
we want privacy to hold for 
the union of any combination of $t$ 
servers' views.
%where the two servers do not communicate.
For example, a 2-server PIR scheme is defined as follows. 



\begin{definition}[Two-server PIR]
A two-server PIR, parametrized with some security parameter $\lambda$, is a 
protocol between a client and two servers %${\sf Server}_{1},
%{\sf Server}_{2}$ 
with the following syntax:  
\begin{itemize}
	\item 
The client's input is $1^\lambda$ and a 
desired index $i \in [n]$, and 
each server's input is a database $\DB \in \{0,1\}^n$.
	\item At the end of the protocol, 
the client outputs a bit $b \in \{0,1\}$.
\end{itemize}
with properties,
\begin{itemize}
	\item \textbf{Correctness}: for all $\lambda, n$, for 
all $\DB \in \{0,1\}^n, i \in [n],$ under honest execution, 
	\[\Pr[b = \DB[i]] = 1%\geq 1 - \mu(n) \tag{negligible $\mu$}
\]
\item \textbf{Privacy}: 
for any $\lambda$, any $n$ that is polynomially bounded in $\lambda$,
any $i,j \in [n]$, any $\DB  \in \{0,1\}^n$, it holds that 
%, and function $view_{s}$ representing the view of server $s$:
	\[{\sf view}_{{1}}(1^\lambda,\DB,i) \approx {\sf view}_{{1}}(1^\lambda,\DB,j)\]
	\[{\sf view}_{{2}}(1^\lambda,\DB,i) \approx {\sf view}_{{2}}(1^\lambda,\DB,j)\]
where ${\sf view}_{{1}}(1^\lambda, \DB, i)$
and ${\sf view}_{{2}}(1^\lambda, \DB, i)$
are random variables representing the 
view of the first and the second server, respectively, in a protocol
execution 
with client input $(1^\lambda, i)$
and server input $(1^\lambda, \DB)$.
\end{itemize}
\end{definition}


Note that PIR schemes can be extended 
for retrieving records containing multiple bits, rather than
just 1-bit records. 


\paragraph{Na\"ive approach.} 
The na\"ive approach is for 
the client to download the entire database. 
However, this approach suffers from linear bandwidth,
and linear server/client computation. 

We will now show some PIR constructions with sublinear bandwidth.


%to retrieve blocks of data by querying the database bit-by-bit.
\section{Single-Server PIR Based on Fully Homorphic
Encryption}

%which is perfect correctness and privacy, but this has linear bandwidth, client computation, and server computation with respect to the database size.

%\subsection{PIR Based on Fully Homomorphic Encryption}
%For a smarter and overall more efficient scheme, 
It is easy to obtain a bandwidth-efficient
PIR scheme if we assume a Fully Homomorphic Encryption (FHE)
scheme. 
%we assume we have an FHE scheme (Gen, Enc, Dec). 
An FHE scheme allows us to perform addition and multiplication operations 
in the ciphertext space. 
%on the ciphers which propagate to their underlying plain-texts.
%\footnote{TA:
%Arithmetic circuit (including addition and multiplication gates) is Turing-complete. Thus, FHE allows the server to evaluate arbitrary computation given the encrypted input. However, we have to be careful about its efficiency -- we usually discuss the time complexity of an algorithm in the RAM model, whereas FHE only works in the circuit model (in general).}
An FHE scheme supports
the following operations: %The scheme is as follows:
\begin{itemize}
\item 
$(\pk, \sk) \leftarrow {\sf Gen}(1^\lambda)$: samples
 a public key $\pk$ and a secret key $\sk$; 
\item 
$c \leftarrow {\sf Enc}(\pk, m)$: encrypts a message $m$
from some message space using the public 
key $\pk$, and outputs the ciphertext $c$;
\item  
$m \leftarrow {\sf Dec}(\sk, c)$: %encrypts a message $m$
decrypts a ciphertext $c$ using the secret key $\sk$,
and outputs a plaintext message $m$;
\item  
$c' \leftarrow {\sf Eval}(\pk, {\sf Circ}, c)$:
given 
the public key $\pk$, some circuit ${\sf Circ}$, and
a ciphertext $c$, output a transformed ciphertext $c'$.
Correctness requires that 
$c'$ be a valid FHE encryption of 
${\sf Circ}(c)$.
\end{itemize}

\paragraph{PIR from FHE.}
We can construct a single-server PIR scheme from an FHE scheme 
as follows.
\begin{enumerate}
	\item The client samples $(\pk,\sk) \leftarrow \FHE.\Gen(1^\lambda),$ and encrypts its 
query as $q \leftarrow \FHE.\Enc(\pk, i)$. The client then sends $q$ to the server.
	\item The server homomorphically evaluates the selection 
circuit $S_{\DB}$ 
by calling 
${c} \leftarrow  \FHE.{\sf Eval}(\pk, S_{\DB}, q)$, 
where $S_{\DB}(i)$ 
is the circuit that selects the $i$-th bit from the input $\DB$, 
%and the circuit is $\tilde{O}(n)$\footnote{$\tilde{O}$ hides the polylogarithmic factors.} in size), 
and sends the resulting ciphertext $c$ back to the client. 
	\item The client decrypts $b \leftarrow \FHE.\Dec(\sk, c)$.
\end{enumerate}
The correctness of the PIR scheme is easy to see
given correctness of the FHE scheme. 
The scheme has $\widetilde{O}(1)$ bandwidth and 
client computation, and $\widetilde{O}(n)$ 
server computation\footnote{We assume a compact
FHE scheme with the following
performance bounds: 
the ciphertext size is 
$\widetilde{O}(1)$ for encrypting a plaintext
of $\widetilde{O}(1)$ bits, and the encryption
and decryption times are also $\widetilde{O}(1)$, 
and the homomorphic evaluation
time
is $\widetilde{O}(|{\sf Circ}|)$
 for a circuit ${\sf Circ}$.} 
where $\widetilde{O}(\cdot)$ hides
${\sf poly}(\lambda, \log n)$ factors.
Note that the server computation is at least linear
because the selection circuit $S_{\DB}(\cdot)$ 
must encode the entire database.

Many recent works optimized FHE-based PIR schemes,  
e.g., Spiral~\cite{spiral} 
and SimplePIR~\cite{simplepir}.

\paragraph{Question:} Can we get sub-linear bandwidth PIR 
without any cryptographic assumptions? 

In fact, this is possible in the two-server setting (we will see this next), 
and later we will prove that it is impossible in the single-server setting.

\section{Two-Server PIR Constructions}
\subsection{$\sqrt{n}$-Bandwidth 
2-Server PIR} % with Information-theoretic (IT) security}
\label{sec:sqrtnbw}

We now introduce a 2-server $\sqrt{n}$-bandwidth
PIR scheme 
with information theoretic security, i.e., 
the scheme does not rely on any cryptographic assumptions.
%Note that IT security implies no harness assumptions, so this is perfect and statistically private.

The key idea is to view the database $\DB \in \{0,1\}^n$ 
as a
$\sqrt{n}$ by $\sqrt{n}$ matrix\footnote{
Without loss of generality, we may assume that $n$ 
is a perfect square --- if not, we can always round it up to the 
nearest perfect square
incurring only constant blowup.}
henceforth denoted
$M \in \{0, 1\}^{\sqrt{n} \times \sqrt{n}}$.
% $S \in \{0,1\}^{\sqrt{n} \times \sqrt{n}}$ matrix. 
Say the client wants to query %the $(i,j)$'th entry. 
the bit $M[i, j]$.
To achieve this, the client will perform
a 2-server PIR protocol
at the  end of which it retrieves
the entire column $M[:, j]$.
%To do this, the server can simply return the entire column $j$, which we can afford to do via the square-root bandwidth. To provide security, the client also supplies a one-hot-vector $q_j$, split into two random shares $v_{1},v_{2}$ via a 2-share secret sharing scheme (that is, sample random $v_1,v_2$ conditioned on $v_1 \oplus v_2 = q_{j}$).
The scheme is as follows:
\begin{enumerate}
\item The client samples random column vectors 
$v_{1},v_{2} \in \{0, 1\}^{\sqrt{n}}$ 
such that 
$v_1 \oplus v_2 = e_j$ where $e_j$
is the vector that is $1$    
at position $j$, and $0$ everywhere else,
and $\oplus$ denotes bitwise XOR (i.e., addition mod 2).
The client sends $v_1$ and $v_2$ 
to the two servers, respectively.
	\item For $i \in \{1, 2\}$, 
%${\sf Server}_i$ 
server $i$ 
computes the following matrix-vector product on receiving $v_i$:
	\[r_{i} \leftarrow M v_{i} \bmod 2, \]
	and sends the $\sqrt{n}$-sized vector $r_{i}$ to the client.
	\item The client computes and outputs $r_{1} \oplus r_{2}$.
\end{enumerate}
It is straightforward to verify correctness by seeing that
\[r_{1} \oplus r_{2} = M v_{1} + M v_{2} \bmod 2 = 
M (v_{1} + v_{2}) \bmod 2 = M \cdot e_j \]
Lastly, this scheme is private, since 
each individual vector $v_{1}$ or $v_{2}$ 
is uniformly random.

\subsection{$n^{\frac{1}{3}}$-Bandwith 2-Server PIR} %\cite{chor1998private}}
Chor et al.~\cite{chor1998private}
showed how to 
get an information-theoretic 2-Server PIR scheme
with $O(n^{1/3})$ bandwidth in expectation.
To get this scheme, we 
will go through a couple stepping stones. 
Specficially, we first 
describe a 2-server scheme 
%We will first motivate this construction with a two server scheme 
with expected $\frac{n}{2}$ bandwidth, 
and an 8-server scheme with $O(n^{\frac 1 3})$ bandwidth. 
Eventually, we will get a 2-server scheme 
with $O(n^{\frac 1 3})$ bandwidth
by coalescing the 
eight servers down to two.

%\begin{enumerate}
\paragraph{Warmup 1: $\mathbb{E}[\frac{n}{2}]$-bandwidth 2-server scheme.}
Below is a simple 2-server PIR scheme with 
$\frac{n}{2}$ expected bandwidth:
\begin{enumerate}
\item  
		The client samples 
a subset $S_{1} \subseteq [n]$ uniformly as follows: for each $i \in [n],$ add $i$ to $S_{1}$ with probability $\frac{1}{2}$. Then, the client computes
		\[S_{2} = S_{1} \Delta \{i\}\]
		where $\Delta$ is the symmetric difference operator, 
that is, if $i$ is included in $S_1$, we remove it from the set, otherwise we add it to the set. 
The client sends $S_{1},S_{2}$ to each server respectively.
\item 
For $i \in \{1, 2\}$, 
server $i$ computes and sends back
		\[r_{i} := \bigoplus_{j \in S_{i}} \DB[j]\]
\item 
		On receiving $r_1,r_2$, the client outputs $r_1 \oplus r_2$. 
\end{enumerate}
	
{Correctness} follows from the fact that $i$ is the only database 
index that appears once in $S_1$ and $S_2$, 
and every other index $j\neq i$ appears twice 
and thus the $j$-th index XORs away.
%$\DB[j] \oplus \DB[j] = 0$
%(hence XOR'ing to 0).
For privacy, 
%		\textbf{Privacy}: 
observe 
that $S_1$ is a uniformly random set.
Further, for any $i \in [n]$, $S_2 = S_1 \Delta \{i\}$ 
is also a uniform random set. To see 
this, 
just consider the following distribution: toss 
$n$ random coins, and then flip the $i$-th coin. 
This distribution is uniformly random. %even when $i$ is known. 
        
\begin{remark}        
Note that if we run the above $n/2$-bandwidth scheme not on bits,
but on blocks of $\sqrt{n}$ size (i.e., treat
the $n$-bit database as $\sqrt{n}$ blocks each of size $\sqrt{n})$, 
the scheme is equivalent to the earlier $\sqrt{n}$-bandwidth scheme
in \Cref{sec:sqrtnbw}.
\end{remark}
	
		%Although this is not a great bandwidth result, it has good ideas for the $n^{\frac 1 3}$-bandwidth case.
\paragraph{Warmup 2: $n^{\frac 1 3}-$bandwidth 8-server scheme.}
We assume that $n = k^3$ 
for some integer $k$ --- if not, we can always
round 
it up to the nearest cubic number, incurring only constant blowup.
For convenience, we will number
the databases indices from $0$ to $n-1$.


The idea is to view the database as a $n^{\frac 1 3} \times n^{\frac 1 3} \times n^{\frac 1 3}$ cube. Then, each index $i \in \{0,1,\dots,n - 1\}$ can be expressed as a triple 
$(x^\star, y^\star, z^\star) \in \{0, \ldots, n^{\frac{1}{3}}-1\}^3$.
Note that 
$(x^\star, y^\star, z^\star)$ is also the base-$n^{\frac{1}{3}}$
representation of $n$. 
%(note we start at $0$ for natural base 3 representation.

The client samples subsets $X,Y,Z \subseteq \{0, \dots, n^{\frac{1}{3}} - 1\}$ independently as follows: for each $x \in \{0,\dots, n^{\frac{1}{3}}\}$ add it to 
$X$ with probability $\frac 1 2$. Do the same for $Y,Z$.
%By the Chernoff bound, the size of 
%each individual set $X$, $Y$ or $Z$
%is $O(n^{1/3})$
Then, the client computes $8$ sets as follows,
where $\times$ denotes Cartesian product: 
%a 3-wise cartesian products with symmetric differences, enumerated as 
		\begin{align*}
			S_{000} &= X \times Y \times Z \\
			S_{001} &= X \times Y \times (Z \Delta \{z^\star\}) \\
			S_{010} &= X \times (Y \Delta \{y^{\star}\}) \times Z \\
			S_{011} &= X \times (Y \Delta \{y^{\star}\}) \times (Z \Delta \{z^\star\}) \\
			S_{100} &= (X \Delta \{x^{\star}\}) \times Y \times Z \\
			S_{101} &= (X \Delta \{x^{\star}\}) \times Y \times (Z \Delta \{z^\star\}) \\
			S_{110} &= (X \Delta \{x^{\star}\}) \times (Y \Delta \{y^{\star}\}) \times Z \\
			S_{111} &= (X \Delta \{x^{\star}\}) \times (Y \Delta \{y^{\star}\}) \times (Z \Delta \{z^\star\}) \\
		\end{align*}

%By Chernoff bound, the size
%of each of these sets is 
%$O(n)$
Although each set's size is linear in $n$ with high probability, 
observe that each of these sets $S_{000}, \ldots, S_{111}$
has a succinct representation 
of size $O(n^{1/3})$ --- for example,
$S_{100}$ can be represented
by the three sets $X \Delta \{x^*\}$, $Y$, and $Z$.

Now, we can construct an $8$-server PIR protocol works as follows:

\begin{enumerate}
\item 
The client samples random $X$, $Y$, $Z$, 
and using its query $(x^*, y^*, z^*)$, it 
computes the 
eight sets $S_{000}, \ldots, S_{111} \subseteq \{0, 1, \ldots, 
n^{\frac13}-1\}$ as mentioned above.

The client sends 
a succinct representation of 
$S_{000}, \ldots, S_{111}$ to each of the eight 
servers, respectively.
\item 
For $i\in \{0, 1, \ldots, 7\}$, server $i$ 
receives $X', Y', Z'$
and computes 
\[p_{i} = \bigoplus_{j \in X' \times Y' \times Z'} \DB[j]\]
It sends back $p_i$ to the client.
\item 
The client computes $p_0 \oplus p_1 \ldots \oplus p_7$.
\end{enumerate}

%See that each of these sets have their sizes concentrated around $\left(\frac{n^{\frac 1 3}}{2}\right)^{3} = \frac n 8$. Then, the client sends a succinct description of each set (i.e., sends the three ``marginal'' vectors instead of the full set) of size $O(n^{\frac 1 3})$ to each server. 


\ignore{		Server $i$ on receiving $S = X' \times Y' \times Z'$ computes 
			\[p_{i} = \bigoplus_{j \in S} \DB[j]\]
		Then the client computes $p_0 \oplus \dots \oplus p_7$. 
}

\begin{claim} 
$p_0 \oplus \dots \oplus p_7 = \DB[i]$ 
where 
$i = (x^\star, y^\star, z^\star)$.
\end{claim}
		\begin{proof}
For every $(x,y,z)$ not equal to the query $(x^*, y^*, z^*)$, 
$\DB[x, y, z]$ appears an even number times in the XOR summation. 
We can pair up the sets to see this. 
On the other hand, the queried index
 $\DB[x^\star, y^\star, z^\star]$ appears only once in the 
XOR summation.
		\end{proof}

		Privacy follows the argument as before, by observing
that each of 
$S_{000}, \ldots, S_{111}$
has the same distribution as:
sample $X$, $Y$, $Z$  
at random 
by including each of $\{0, \ldots, n^{\frac13}-1\}$
with probability $1/2$, 
and output the Cartesian product $X \times Y \times Z$.
%\end{enumerate}

\paragraph{$n^{\frac13}$-bandwidth 2-server scheme.}
Now, we finally compress this scheme from eight servers to two servers. To do this, the client sends $S_{000}$ to server 1 and $S_{111}$ to server 2.

Henceforth, 
the parity for some set $S$
means $\oplus_{j \in S} \DB[j]$. 
Each server, on receiving $X', Y', Z'$, 
returns not only the parity of the set 
$X' \times Y' \times Z'$, but also 
the parities of the following 
$3 n^{\frac1 3}$ sets:  
\begin{enumerate}
\item For each $x' \in \{0,\dots,n^{\frac 1 3} - 1\}$, the server
returns 
the parity of the 
set $(X' \Delta \{x'\}) \times Y' \times  Z'$. %as in the previous scheme.
	\item For each $y' \in \{0,\dots,n^{\frac 1 3} - 1\}$, the server 
returns the parity of  
the set $X' \times (Y' \Delta \{y'\}) \times  Z'$.
	\item For each $z' \in \{0,\dots,n^{\frac 1 3} - 1\}$, the server
returns the parity of the set $X' \times Y' \times  (Z' \Delta \{z'\})$.
\end{enumerate}

Now, observe that 
the client can find 
the parities of $S_{000}$ and $S_{100}, S_{010}, S_{001}$
from the first server's response, 
and it can find the parities 
of $S_{111}$ and $S_{011}, S_{101}, S_{110}$
from the second server's response.
Thus, it can still reconstruct the answer $\DB[i]$
as in the eight-server scheme. 


%$1+3n^{1/3}$ parities to the client.
\ignore{
That is, the server 1 will actually compute $S_{000}$ and $S_{100}, S_{010}, S_{001}$ will be in those $3n^{1/3}$ parities. 
Similarly, the server 2 will compute $S_{111}$, and $S_{011},S_{101},S_{110}$ will be in those $3n^{1/3}$ parities.
The client will be able to pick out the correct parities corresponding to its actual query.
}

Thus, this scheme still has $O(n^{\frac 1 3})$ bandwidth, 
and correctness and privacy follow as before.

\subsection{Status Quo and Open Problems}
It is known 
any 2-server information-theoretic PIR must incur
 at least 
$5 - o(1)\log n$ bandwidth~\cite{WdW05},  
while the best known upper bound requires 
	$n^{O\left(\sqrt{\lg \lg n / \lg n}\right)} = n^{o(1)}$ bandwith (i.e., 
sub-polynomial
bandwidth)~\cite{dvir20162}. 
Closing this gap is a long-standing open problem. 


