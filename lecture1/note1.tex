
%\section{Lecture 1: Private Information Retrieval}
The Private Information Retrieval problem was first introduced by Chor, Kushilevitz, Goldreich and Sudan~\cite{chor1998private}.
In this setting, we will have a client and 
one or more server(s). The servers each have a public database indexed from $1$ to $n$ (e.g., the DNS repository, 
a repository of webpages, a leaked password database, etc).

A client wants to fetch an entry 
indexed $i \in [n]$ 
from this database 
but does not want to leak its query to the server(s). More formally,
we define a single-server PIR scheme as follows.

\begin{definition}[Single-server PIR]
A single-server PIR, parametrized
by a security parameter $\lambda \in \mathbb{N}$, is a 
protocol between a client and a server with the following
syntax: 
\begin{itemize}
	\item The client's input is a desired 
index $i \in [n]$, and the server's input is a database $\DB \in \{0,1\}^n$.
Both the client and server also obtain $1^\lambda$ as input. 
	\item At the end of the protocol, the client outputs a bit $b \in \{0,1\}$.
\end{itemize}

We want the scheme to satisfy the following properties.
\begin{itemize}
\item \textbf{Correctness}: for all $\lambda, n$, 
%any $n$ that is polynomially bounded in $\lambda$, 
for any $\DB \in \{0,1\}^n, i \in [n],$ under honest execution, 
	\[\Pr[b = \DB[i]] = 1 %- \mu(n) %\tag{negligible $\mu$}
\]
	
\item \textbf{Privacy}: For any $\lambda$, 
any $n$ polynomially bounded in $\lambda$, any 
$i,j \in [n], \DB  \in \{0,1\}^n$, 
it holds that  
%and function $view_{s}$ representing the view of server $s$:
	\[{\sf view}_s(1^\lambda, \DB,i) \approx {\sf view}_s(1^\lambda, \DB,j)\]
where ${\sf view}_{s}(1^\lambda, \DB, i)$ 
is a random variable representing the view of the server  
if we execute the PIR protocol 
over client input $(1^\lambda, i)$ and server input $(1^\lambda, \DB)$, 
and $\approx$ stands for 
statistical or computational indistinguishability.
%identically distributed, statistically, 
%or computationally indistinguishable.
%representing the view of server $s$:
\end{itemize}
\end{definition}
%Intuitively, the privacy definition means that from the server's view, it cannot tell what the user is querying. One can also define a definition for malicious privacy, but since our schemes of interest are one-round, we do not make this distinction.

\begin{remark}[Honest-server vs. malicious-server privacy]
The above privacy definition assumes an honest server.
It is also possible to define privacy against a malicious server.
In today's lecture, all the PIR constructions
will only have a single round-trip --- 
in this special case, honest-server privacy
and malicious-server
privacy are equivalent. So we will simply define honest-server privacy here.
\end{remark}



This definition is naturally extended to 
a setting with two or more servers that do not communicate, 
where privacy should hold for any individual server's view. 
In a setting with more than two servers, it also makes
sense to define $t$-out-of-$n$ security, where
we want privacy to hold for 
the union of any combination of $t$ 
servers' views.
%where the two servers do not communicate.




\begin{definition}[Two-server PIR]
A two-server PIR, parametrized with some security parameter $\lambda$, is a 
protocol between a client $c$ and two servers $s_{1},s_{2}$ 
with the following syntax:  
\begin{itemize}
	\item 
The client's input is $1^\lambda$ and a 
desired index $i \in [n]$, and 
each server's input is a database $\DB \in \{0,1\}^n$.
	\item At the end of the protocol, 
the client outputs a bit $b \in \{0,1\}$.
\end{itemize}
with properties,
\begin{itemize}
	\item \textbf{Correctness}: for all $\lambda, n$, for 
all $\DB \in \{0,1\}^n, i \in [n],$ under honest execution, 
	\[\Pr[b = \DB[i]] = 1%\geq 1 - \mu(n) \tag{negligible $\mu$}
\]
	
	\item \textbf{Privacy}: $\forall i,j \in [n], \DB  \in \{0,1\}^n$, and function $view_{s}$ representing the view of server $s$:
	\[view_{s_{1}}(n,\DB,i) \cong view_{s_{1}}(n,\DB,j)\]
	\[view_{s_{2}}(n,\DB,i) \cong view_{s_{2}}(n,\DB,j)\]
\end{itemize}
\end{definition}


Note that PIR schemes can additionally be extended to retrieve blocks of data by querying the database bit-by-bit.
\section{Single Server PIR constructions}
Naively, the client can just download the entire database, which is perfect correctness and privacy, but this has linear bandwidth, client computation, and server computation with respect to the database size.

\subsection{FHE PIR}
For a smarter and overall more efficient scheme, we assume we have an FHE scheme (Gen, Enc, Dec). An FHE scheme allows us to perform addition and multiplication operations on the ciphers which propagate to their underlying plain-texts.
\footnote{TA:
Arithmetic circuit (including addition and multiplication gates) is Turing-complete. Thus, FHE allows the server to evaluate arbitrary computation given the encrypted input. However, we have to be careful about its efficiency -- we usually discuss the time complexity of an algorithm in the RAM model, whereas FHE only works in the circuit model (in general).}

The scheme is as follows:
\begin{enumerate}
	\item The client samples $(pk,sk) \leftarrow \FHE.\Gen(1^\lambda),$ and encrypts their query as $m \leftarrow \FHE.\Enc(i)$. The client then sends $m$ to the server.
	\item The server homomorphically evaluates the selection circuit $S$ and let ${c} = {\sf Eval}(S, m)$ ($S(i)$ selects the $i$-th bit from $\DB$ and the circuit is $\tilde{O}(n)$\footnote{$\tilde{O}$ hides the polylogarithmic factors.} in size), and sends it to the client. 
	\item The client decrypts $\FHE.\Dec(sk,c)$.
\end{enumerate}
The correctness of the scheme is clear. The scheme  has $\tilde{O}(1)$ bandwidth and client computation and $\tilde{O}(n)$ server computation. Some notable PIR schemes include Spiral~\cite{spiral} and SimplePIR~\cite{simplepir}.

\textbf{Question}: Can we get sub-linear bandwidth without any cryptographic (hardness) assumptions? In fact, this is possible in the two server setting (we will see this next), and we can prove that this is impossible in the one server case.

\section{Two Server PIR Constructions}
\subsection{$\sqrt{n}$-BW 2-Server-Scheme with Information-theoretic (IT) security}
Note that IT security implies no harness assumptions, so this is perfect and statistically private.

The key idea is the view of database $DB \in \{0,1\}^n$ as a $S \in \{0,1\}^{\sqrt{n} \times \sqrt{n}}$ matrix. Say the client wants to query the $(i,j)$'th entry. To do this, the server can simply return the entire column $j$, which we can afford to do via the square-root bandwidth. To provide security, the client also supplies a one-hot-vector $q_j$, split into two random shares $v_{1},v_{2}$ via a 2-share secret sharing scheme (that is, sample random $v_1,v_2$ conditioned on $v_1 \oplus v_2 = q_{j}$).

The scheme is as follows:
\begin{enumerate}
	\item The client creates $v_{1},v_{2}$ as described above. The client sends these shares to servers $s_{1},s_{2}$ respectively.
	\item Each server $s_{i}$, on receiving $v_{i}$ computes the matrix-vector product,
	\[r_{i} \leftarrow S v_{i} \bmod 2\]
	and sends the $\sqrt{n}$-size vector $r_{i}$ to the client.
	\item The client computes and outputs $r_{1} \oplus r_{2}$.
\end{enumerate}
It is straightforward to verify correctness by seeing that
\[r_{1} \oplus r_{2} = S r_{1} \oplus Sr_{2} \bmod 2 = S (r_{1} \oplus r_{2}) \bmod 2\]
Lastly, this scheme is private, since $v_{1},v_{2}$ are uniformly random in each server's view, respectively.

\subsection{$n^{\frac{1}{3}}$-BW 2-server Scheme\cite{chor1998private}}
We will first motivate this construction with a two server scheme with expected $\frac{n}{2}$ bandwidth and a eight server scheme with $n^{\frac 1 3}$ bandwidth. 
\begin{enumerate}
	\item \textbf{$\mathbb{E}[\frac{n}{2}]$-BW 2-server Scheme}
	
		The client samples $S_{1} \subseteq [n]$ uniformly as follows: for each $i \in [n],$ add $i$ to $S_{1}$ with probability $\frac{1}{2}$. Then, the client computes
		\[S_{2} = S_{1} \Delta \{i\}\]
		where $\Delta$ is the symmetric difference operator. That is, if $i$ is included in $S_1$, we remove it from the set, otherwise we add it to the set. $S_{1},S_{2}$ are sent to each server respectively, where server $i$ computes and sends back,
		\[r_{i} \leftarrow \bigoplus_{j \in S_{i}} \DB[j]\]
		The client on receiving $r_1,r_2$ outputs $r_1 \oplus r_2$. 
	
		\textbf{Correctness} follows from the fact that $i$ is the only database index that appears once, with every other index appearing twice (hence XOR'ing to 0).

		\textbf{Privacy}: First, $S_1$ is uniformly random. Second, to see why $S_2$ is uniformly random to server 2, just consider the following distribution -- ``tossing $n$ random coins, and flip the $i$-th coin afterwards, regardless of the original result''. This distribution is uniformly random even when $i$ is known. % because we are just flipping the random coin for $$
        
\begin{remark}        
Note that if we run the above $n/2$-BW scheme not on bits,
but on blocks of $\sqrt{n}$ size (i.e., treat
the $n$-bit database as $\sqrt{n}$ blocks each of size $\sqrt{n})$, 
the scheme is equivalent to the earlier $\sqrt{n}$-BW scheme.
\end{remark}
	
		%Although this is not a great bandwidth result, it has good ideas for the $n^{\frac 1 3}$-bandwidth case.
	\item \textbf{$n^{\frac 1 3}-$BW 8-server Scheme}
	
		The idea is to view the database as a $n^{\frac 1 3} \times n^{\frac 1 3} \times n^{\frac 1 3}$ cube. Then, each index $i \in \{0,1,\dots,n - 1\}$ can be expressed as a thruple $(x^\star, y^\star, z^\star)$ (note we start at $0$ for natural base 3 representation.

		The client samples $X,Y,Z \subseteq \{0, \dots, n^{\frac{1}{3}} - 1\}$ independently as follows: for each $x \in \{0,\dots, n^{\frac{1}{3}}\}$ add it to X with probability $\frac 1 2$. Do the same for $Y,Z$.
	
		Then, we compute $8$ different sets by a 3-wise cartesian products with symmetric differences, enumerated as 
		\begin{align*}
			S_{000} &= X \times Y \times Z \\
			S_{001} &= X \times Y \times (Z \Delta \{z^\star\}) \\
			S_{010} &= X \times (Y \Delta \{y^{\star}\}) \times Z \\
			S_{011} &= X \times (Y \Delta \{y^{\star}\}) \times (Z \Delta \{z^\star\}) \\
			S_{100} &= (X \Delta \{x^{\star}\}) \times Y \times Z \\
			S_{101} &= (X \Delta \{x^{\star}\}) \times Y \times (Z \Delta \{z^\star\}) \\
			S_{110} &= (X \Delta \{x^{\star}\}) \times (Y \Delta \{y^{\star}\}) \times Z \\
			S_{111} &= (X \Delta \{x^{\star}\}) \times (Y \Delta \{y^{\star}\}) \times (Z \Delta \{z^\star\}) \\
		\end{align*}
		See that each of these sets have their sizes concentrated around $\left(\frac{n^{\frac 1 3}}{2}\right)^{3} = \frac n 8$. Then, the client sends a succinct description of each set (i.e., sends the three ``marginal'' vectors instead of the full set) of size $O(n^{\frac 1 3})$ to each server. 

		Server $i$ on receiving $S = X' \times Y' \times Z'$ computes 
			\[p_{i} = \bigoplus_{j \in S} \DB[j]\]
		Then the client computes $p_0 \oplus \dots \oplus p_7$. 

		We claim that $p_0 \oplus \dots \oplus p_7 = \DB[i], i = (x^\star, y^\star, z^\star)$.
		\begin{proof}
		For every $(x,y,z)$ not equal to the query, it will appear an even number times in the summation. We can pair up the sets to see this. 

		On the other hand, $(x^\star, y^\star, z^\star)$ only appears once in the summation so we are done.
		\end{proof}

		Privacy follows the argument as the previous case. 
\end{enumerate}

Now, we finally compress this scheme from eight servers to two servers. To do this, the client sends $S_{000}$ to server 1 and $S_{111}$ to server 2.

Each server on receiving $X' \times Y' \times Z'$ calculates $3 n^{\frac1 3}$ parities to send to the client as follows:
\begin{enumerate}
	\item For each $x' \in \{0,\dots,n^{\frac 1 3} - 1\}$ we calculate the parity for $(X' \Delta \{x'\}) \times Y' \times  Z'$ as in the previous scheme.
	\item For each $y' \in \{0,\dots,n^{\frac 1 3} - 1\}$ we calculate the parity for $X' \times (Y' \Delta \{y'\}) \times  Z'$ as in the previous scheme.
	\item For each $z' \in \{0,\dots,n^{\frac 1 3} - 1\}$ we calculate the parity for $X' \times Y' \times  (Z' \Delta \{z'\})$ as in the previous scheme.
\end{enumerate}

Now each server returns $1+3n^{1/3}$ parities to the client.
That is, the server 1 will actually compute $S_{000}$ and $S_{100}, S_{010}, S_{001}$ will be in those $3n^{1/3}$ parities. 
Similarly, the server 2 will compute $S_{111}$, and $S_{011},S_{101},S_{110}$ will be in those $3n^{1/3}$ parities.
The client will be able to pick out the correct parities corresponding to its actual query.

Thus, this scheme still has $n^{\frac 1 3}$ bandwidth, and correctness and security still follow.

\section{State of the Art and Open Problem}
	For the 2-server setting, the best
known lower bound states that $5 - o(1)\log n$ bandwidth is necessary (\cite{WdW05}),  while the best known upper bound requires 
	$n^{O\left(\sqrt{\lg \lg n / \lg n}\right)} = n^{o(1)}$, i.e., 
sub-polynomial
bandwidth (\cite{dvir20162}). Closing this gap is a long-standing open problem. 


